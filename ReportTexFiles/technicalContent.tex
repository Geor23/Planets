\chapter{Technical Content}

%Choose some of the more interesting/challenging/novel aspects of the project to discuss in detail. Explain why they are technically interesting or challenging and discuss how the problems were overcome. A successful game will:

%Have several technically interesting challenges which have been overcome, and explained succinctly, including good software development.

%Be playable with little or no training - but have enough depth to satisfy an experienced gamer.

%Have come from a development team who have demonstrated a willingness to address problems in the team and have found solutions to them.

%Have a relevant look and feel (creative content creation) and be novel.

%All the above will be assessed with equal weighting.

\section{Technical Challenges}
\markboth{\MakeUppercase{\thechapter}}
Creating a server where players will be able to connect in order to play. We had to develop our own dedicated server, using an Intel NUC in order to provide our players a dedicated network in which they could enjoy our game.\\

%Turning the game into a multiplayer experience over a network: One of the biggest obstacles our team had to face in order for our game to take the desired form. Sharing resources over the network was quite challenge since passing the various game updates over a network like who killed who, which resources have been picked up or which have spawn, cannot be passed simply as a variable, its a much more complex procedure. We had to take into consideration the amount of data we had to move send around the network, as we wanted to contain that as much as possible.\\

One of the major challenges for us was making the game a fully networked game, allowing for many players to play together for their own mobile devices. We require for a fair amount of data to be sent and processed by both the servers as well as the player clients and observer clients. Movement, rotation, deaths, kills, scores and game updates all needed to be synchronised and updated for all players. We initially started by using the Unity Networking structures available to us, which allowed for us to establish the basic functionality for the game in good time. Once the basic plan layout had been established, we were able to test the game in a basic form, which was helpful in assessing the current gameplay and how to improve on it. \\

We realised at this stage two things: firstly that we needed to focus more of the game on the observer view, and that the Unity Networking structures we'd used were far from optimal for our usage. We from here worked on replacing the Unity structures of our own, allowing us to create very specific networking functionality for our game with single message passes between only the required devices. By moving the a lot of the computation to the observer displays, we were able to reduce the amount of network traffic significantly. By making the players only visible on the observer view, we reduced the need for all clients to recieve constant updates as to player actions, only needing for the observer to be updated with these changes. Everything besides deaths could then be dealt with on the observers, be it interactions with resources, events, or scores. Even events which seemingly would require some network traffic, such as the rotation of player objects, was with some thought and mathematics able to be purely calculated on the observer. \\

Porting of the game to the planetarium dome: The way planetarium projectors work is that they project the computer monitors onto a large dome. In order to cope with this we had to rescale all our assets and prefabs.\\

Utilize the 3D capabilities of the Planetarium for our games purposes:

% ------------------------------------------------------------------------


%%% Local Variables: 
%%% mode: latex
%%% TeX-master: "../thesis"
%%% End: 
